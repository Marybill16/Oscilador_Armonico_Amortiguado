% Teoría del oscilador no lineal amortiguado
% Archivo: teoria.tex

\documentclass[12pt]{article}
\usepackage{amsmath, amssymb}
\usepackage{physics}
\usepackage{graphicx}
\usepackage{geometry}
\geometry{margin=2.5cm}

\title{Teoría del Oscilador No Lineal Amortiguado}
\author{}
\date{}

\begin{document}

\maketitle

\section*{1. Introducción}
En este documento se presenta la teoría utilizada para modelar un oscilador no lineal sometido a una fuerza restauradora proporcional al desplazamiento y una fuerza de amortiguamiento cuadrática proporcional al cuadrado de la velocidad.

El sistema está gobernado por la ecuación diferencial:
\begin{equation}
 m \ddot{x} + \lambda \, \dot{x} |\dot{x}| + kx = 0,
\end{equation}
donde $m$ es la masa, $k$ es la constante de resorte y $\lambda$ el coeficiente de amortiguamiento no lineal.

\section*{2. Conversión a un Sistema de Primer Orden}
Para aplicar el método de Runge--Kutta 4 (RK4), convertimos la ecuación de segundo orden en un sistema de ecuaciones de primer orden.

Definimos:
\begin{align}
 y_1 &= x, \\
 y_2 &= \dot{x}.
\end{align}

Las ecuaciones quedan:
\begin{align}
 \dot{y}_1 &= y_2, \\
 \dot{y}_2 &= -\frac{k}{m} y_1 - \frac{\lambda}{m} y_2 |y_2|.
\end{align}

\section*{3. Método de Runge--Kutta de Orden 4}
Para integrar un sistema de ecuaciones:
\begin{equation}
 \dot{\mathbf{y}} = \mathbf{f}(t, \mathbf{y}),
\end{equation}
se calcula:

\begin{align}
 \mathbf{k}_1 &= \mathbf{f}(t, \mathbf{y}), \\
 \mathbf{k}_2 &= \mathbf{f}\left(t + \frac{h}{2},\, \mathbf{y} + \frac{h}{2} \mathbf{k}_1\right), \\
 \mathbf{k}_3 &= \mathbf{f}\left(t + \frac{h}{2},\, \mathbf{y} + \frac{h}{2} \mathbf{k}_2\right), \\
 \mathbf{k}_4 &= \mathbf{f}\left(t + h,\, \mathbf{y} + h \mathbf{k}_3\right).
\end{align}

Luego:
\begin{equation}
 \mathbf{y}(t+h) = \mathbf{y}(t) + \frac{h}{6} \left( \mathbf{k}_1 + 2\mathbf{k}_2 + 2\mathbf{k}_3 + \mathbf{k}_4 \right).
\end{equation}

\section*{4. Amortiguamiento No Lineal}
El término disipativo:
\begin{equation}
 F_d = -\lambda v |v|,
\end{equation}
corresponde a amortiguamiento cuadrático, típico en sistemas con resistencia aerodinámica o fluidos. A diferencia del amortiguamiento lineal ($F = -b v$), aquí la disipación crece más rápidamente con la velocidad.

\section*{5. Conclusión}
El método RK4 permite integrar de manera precisa sistemas no lineales como el oscilador amortiguado cuadráticamente. Esta teoría justifica las ecuaciones implementadas en el código y describe la dinámica física del sistema.

\end{document}
