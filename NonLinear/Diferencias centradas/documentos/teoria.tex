\documentclass[a4paper,10pt]{article}
\usepackage[utf8]{inputenc}
\usepackage{amsmath,amssymb,graphicx}
\usepackage{xcolor}
\usepackage{geometry}
\geometry{margin=2cm}

\title{Simulación de una Partícula en una Caja 1D con Pared Móvil}
\author{John Hernán Díaz Forero}
\date{Octubre 2025}

\begin{document}
	\maketitle
	
	\section{Planteamiento del problema}
	Una partícula de masa \(m\) se mueve dentro de una caja unidimensional.
	La pared derecha se desplaza con velocidad constante \(u\):
	\[
	x_R(t) = L_0 - ut
	\]
	
	\section{Condiciones de frontera}
	\[
	\begin{cases}
		x < 0 &\Rightarrow v \to -v, \\
		x > x_R(t) &\Rightarrow v \to -2u - v.
	\end{cases}
	\]
	
	\section{Solución numérica}
	Se integra paso a paso:
	\[
	x_{n+1} = x_n + v_n \Delta t
	\]
	y se registran \(t, x, v, x_R\).
	
	\section{Resultados}
	El gráfico muestra la trayectoria temporal y la animación ilustra la evolución
	dinámica de la partícula dentro de la caja móvil.
	
	\section{Conclusión}
	El modelo conserva la energía cinética salvo durante el choque con la pared
	en movimiento, donde el marco de referencia cambia.
	
\end{document}
