\documentclass[a4paper,10pt]{article}
\usepackage[utf8]{inputenc}
\usepackage{amsmath,amssymb,graphicx}
\usepackage{xcolor}
\usepackage{geometry}
\usepackage{float}
\geometry{margin=2cm}

\title{Montaje Experimental de Movimiento Armónico Amortiguado}
\author{López N., Vitola J., Velandia M., Riaño L.}
\date{Diciembre 2025}

\begin{document}
	\maketitle
	
	\section{Montaje Experimental}
	Soporte universal con un resorte posicionado en vertical con una masa $m$ en el extremo inferior. La masa se encuentra sumergida en un recipiente con agua, dispuesta de tal manera que no salga del mismo durante las oscilaciones.
	
	\begin{figure}[H]
		\centering
		\includegraphics[scale=0.2]{montaje.png}
		\caption{\centering Montaje experimental realizado.}
		\label{1}
	\end{figure}
	
	Al fondo se dispone de una regla que sirve como referencia de calibración para VideoTracker, programa del cual se obtuvieron los datos para el archivo de entrada (posición y tiempo).
	
	\section{Condiciones Iniciales y Datos Conocidos}
	\[
	\begin{cases}
		x(t)=-0.052195650(m) & \Rightarrow t=0(s) , \\
		\frac{dx(t)}{dt}=v(t)=0(m/s) & \Rightarrow t=0(s)
	\end{cases}
	\]
	
	En cuanto a las características conocidas de los implementos tenemos que:
	\begin{align*}
		m &= 0.145 kg \\
		k &= 2.9686 N/m \\
		\lambda &= 0.1718 s^{-1}
	\end{align*}
	
	\section{Solución numérica}
	Se uso derivación numérica por diferencias centradas para hallar las velocidades punto a punto, y con estas se halló el momentum.
	\begin{align*}
		v_{i} &= \frac{x_{i+1}-x_{i-1}}{t_{i+1}-x_{i-1}} \\
		\rho_i &= v_{i}*m
	\end{align*}
	
	
	\section{Resultados}
	El código genera un gráfico con la evolución temporal de la posición y la velocidad, y otro con el momentum en función de la posición. Genera también un .dat con 4 columnas: t, x, v, $\rho$
	
	\section{Conclusión}
	En la gráfica se observa un movimiento oscilatorio, cuya amplitud disminuye al avanzar el tiempo, indicando una disminución en la altura máxima alcanzada por la masa y un aumento en su altura mínima.
	
	Ligado a esto se observa que la velocidad describe un comportamiento similar, dando cuenta del frenado rápido de la masa debido al agua del recipiente. 
	
	Para momentum contra posición, se observa la trayectoria de espiral, que da cuenta de la pérdida de energía del sistema y por tanto la disminución de las amplitudes.
	
\end{document}
